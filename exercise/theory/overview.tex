%====================================================
\section{Overview}
%====================================================

The goal of this exercise is to get you started with debugging using tools such as \verb*|gdb| and
\verb*|valgrind|. To this end, a toy code is provided, both in C and in Fortran, which has been
intentionally riddled with bugs for you to find and fix. The buggy code is stored in
\verb|exercise/src/c| and \verb*|exercise/src/f90|, respectively. Solutions are available in
\verb|exercise/solutions/|. You may want to run the codes stored there to compare what the correct
output should be.


This overview aims to provide a brief summary of the toy code, its main algorithm, and the equations
being solved. Much more information is given in subsequent Sections, but this overview should
suffice to get you started.


The toy code solves an equation called ``\textit{linear advection with constant coefficients}'' in
1D. That equation is given by:

\begin{equation}
	\DELDT{ q } + a \cdot \DELDX{q} = 0 \label{eq:advection-equation-overview}
\end{equation}

with $a = \text{ const } > 0$. An analytical solution exists, and is given by any function $q(x,t)$
which satisfies

\begin{equation}
	q(x, t) = q(x - a (t - t_0), t_0) .\label{eq:advection-solution-overview}
\end{equation}

It describes a function $q(x, t)$ of any arbitrary shape being translated by $a t$ as time evolves
(see e.g. Figure~\ref{fig:advection-analytical-solution}). So the solution you should be expecting
is the same initial shape traveling along the $x$-axis with constant speed $a$, save for some
numerical diffusion (see Fig.~\ref{fig:linear-advection-results}).

To solve the problem numerically, we discretize space in $N$ cells of equal width $\Delta x$. The
cells are given an integer index $i$. The numerical solution at a subsequent time step $t^{n+1}$
given the current state $t^n$ is given by:

\begin{equation}
	q_i^{n+1} = q_i^{n} +  a \frac{\Delta t}{\Delta x} \left( q_{i-1}^n - q_{i}^n \right)
	\label{eq:advection-numerical-overview}
\end{equation}

where the time step size $\Delta t$ must be limited by the so-called ``CFL condition''

\begin{equation}
	\Delta t_{max} = C_{cfl} \frac{\Delta x}{a} \label{eq:CFL1D-overview}
\end{equation}

where $C_{cfl} \in [0, 1) $ is a user-set factor.

For this exercise, we employ \emph{periodic boundary conditions}, i.e. we set up the solver such
that whatever leaves or enters on one side of our simulation box also enters or leaves on the
opposite side, respectively. To achieve this behavior, we add additional cells (``\emph{ghost
cells}'') around our simulation box.

Suppose we have 1D grid with $N$ cells and require 2 ghost cells each, which will have indices $-2$,
$-1$, $N+1$, and $N+2$. We obtain periodic boundary conditions by enforcing

\begin{align*}
	q_{-2} &= q_{N-1} \\
	q_{-1} &= q_{N}	\\
	q_{N+1} &= q_0 	\\
	q_{N+2} &= q_1
\end{align*}

See also Figure~\ref{fig:boundary_periodic}.


Finally, a brief description of the full algorithm of the solver:

\begin{tcolorbox}
To start, set $t_{current} = t^0 = t_{start}$ and set up the initial states $q_i^0$ for
each cell $i$.\\[.5em]
%
While $t_{current} < t_{end}$, solve the $n$-th time step:\\[.5em]
%
\indent~~~~Apply the periodic boundary conditions \\[.5em]
%
\indent~~~~Find the maximal permissible time step $\Delta t$ using eq.~\ref{eq:CFL1D-overview}
\\[.5em]
%
\indent~~~~For each cell $i$, find the updated states $q_i^{n+1}$ using
eq.~\ref{eq:advection-numerical-overview} \\[.5em]
%
\indent~~~~Update the current time: $t_{current} = t^{n+1} = t^n + \Delta t$

\end{tcolorbox}

\newpage