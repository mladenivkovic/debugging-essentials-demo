\documentclass[10pt, a4paper, english, parskip, twocolumn]{scrartcl}

\usepackage{color, graphicx, float}
\usepackage[utf8]{inputenc}
\usepackage[english]{babel}

\usepackage{geometry}
\geometry{margin=1cm}

\usepackage[document]{ragged2e}
\usepackage{listings}

\lstdefinestyle{myCodeStyle}{
    backgroundcolor=\color{ashgrey!20},
%     commentstyle=\color{codegreen},
%     keywordstyle=\color{magenta},
%     numberstyle=\tiny\color{codegray},
%     stringstyle=\color{codepurple},
    basicstyle=\ttfamily,
    breakatwhitespace=false,
    breaklines=true,
    captionpos=b,
    keepspaces=true,
    numbers=none,
    numbersep=5pt,
    showspaces=false,
    showstringspaces=false,
    showtabs=false,
    tabsize=2
}
\lstset{style=myCodeStyle}



%-----------------------------------------------
% FORMAT TITLE
%-----------------------------------------------

% Set fonts of document parts
\setkomafont{title}{\rmfamily\bfseries\boldmath}
\addtokomafont{section}{\rmfamily\bfseries\boldmath}
\addtokomafont{subsection}{\rmfamily\bfseries\boldmath}
\addtokomafont{subsubsection}{\rmfamily\bfseries\boldmath}
\addtokomafont{disposition}{\rmfamily} % table of contents and stuff
\setkomafont{descriptionlabel}{\rmfamily\bfseries\boldmath}






%-----------------------------------------------
% TEXT IN BOXES
%-----------------------------------------------

\usepackage[framemethod=TikZ]{mdframed}

% New Colors (needs to be after usepackage mdframed)
\definecolor{babyblueeyes}{rgb}{0.63, 0.79, 0.95}
\definecolor{ashgrey}{rgb}{0.7, 0.75, 0.71}
\definecolor{caribbeangreen}{rgb}{0.0, 0.8, 0.6}
\definecolor{bittersweet}{rgb}{1.0, 0.44, 0.37}

\mdfdefinestyle{section}{%
       %rightline=true,
       innerleftmargin=10,
       innerrightmargin=10,
       frametitlerule=false,
%        frametitlerulecolor=black,
%        frametitlerulewidth=2
}





\newcommand{\code}[1]{\texttt{#1}}
\newcommand{\ddash}{-{}-}
\newcommand{\gdbsection}[1]{\begin{mdframed}[frametitlebackgroundcolor=bittersweet,style=section,
frametitle=#1]\end{mdframed}}



\pagestyle{empty}


\begin{document}


\section*{\code{gdb} cheatsheet}




\gdbsection{running gdb}
%
Remember to compile your program with debugging symbols enabled!
That's usually the \code{-g} or \code{-ggdb} flag.

%
\begin{lstlisting}[belowskip=0pt]
$ gdb /path/to/your/executable
\end{lstlisting}
%
\quad Running your executable with \code{gdb}
%
\begin{lstlisting}[belowskip=-0.8 \baselineskip]
$ gdb --args /path/to/your/executable \
  --arg1 --arg2
\end{lstlisting}
%
\quad or
%
\begin{lstlisting}[belowskip=-0.8 \baselineskip]
$ gdb /path/to/your/exec
(gdb) run --args --arg1 --arg2
\end{lstlisting}
\quad If \code{exec} requires cmdline args (\code{\ddash arg1, \ddash arg2})





\gdbsection{gdb basic commands}
%
{\renewcommand{\arraystretch}{1.15}%
% \small%
\begin{table}[H]
\begin{center}
\begin{tabular}{p{2.8cm}p{5.4cm}}
\code{r (run)}          & run the program \code{gdb} loaded \\
\code{c (continue)}     & continue run\\
\code{n (next)}         & execute next line\\
\code{s (step)}         & execute next step\\
\code{finish}           & after a breakpoint, run function until end and halt there\\[1em]
%
\code{l (list)}         & show source code at current loc\\
\code{l <lnr>}          & show source code at line $<$lnr$>$\\
\code{tui enable}       & start a fancy Text User Interface\\[1em]
%
\code{p (print) <var>}  & print variable $<$var$>$\\
\code{p *<var\_p>}      & print value of pointer $<$var\_p$>$ instead of address\\
\code{display <var>}    & print variable $<$var$>$ every time it is touched throughout the run\\
\code{info locals}      & print all local variables\\[1em]
%
\code{b (break) <loc>}  & set a breakpoint at $<$loc$>$\\
\code{tbreak <loc>}     & set a temporary breakpt at $<$loc$>$\\
\code{watch <var>}      & set a watchpoint at var $<$var$>$\\[1em]
%
\code{bt (backtrace)}   & show stack trace\\
\code{where}            & show current location in trace\\
\code{frame}            & show current location in trace\\
\code{frame <nr>}       & change into frame $<$nr$>$
\end{tabular}
\end{center}
\end{table}
}












\gdbsection{gdb and core dumps}

\begin{lstlisting}[belowskip=-0.8 \baselineskip]
$ ulimit -S -c unlimited
\end{lstlisting}
\quad To enable core dumps on linux

\begin{lstlisting}[belowskip=-0.8 \baselineskip]
$ coredumpctl info
\end{lstlisting}
\quad Show info of last coredump.

\begin{lstlisting}[belowskip=-0.8 \baselineskip]
$ gdb -c core.XXXX path/to/executable
\end{lstlisting}
\quad Load core dump with \code{gdb}

\begin{lstlisting}[belowskip=-0.8 \baselineskip]
$ coredumpctl debug [--debugger=/path/to/gdb]
\end{lstlisting}
\quad launch \code{gdb} through \code{coredumpctl}










\gdbsection{gdb breakpoints}

\begin{lstlisting}[belowskip=-0.8 \baselineskip]
break <line number in main file>
\end{lstlisting}
%
\quad To set a breakpoint (\textit{before} you execute \code{run})
%
\begin{lstlisting}[belowskip=-0.8 \baselineskip]
break path/to/file.c:<line_nr>
\end{lstlisting}
\quad To set a breakpoint on $<$line\_nr$>$ in specific file
%
\begin{lstlisting}[belowskip=-0.8 \baselineskip]
break file.c:function_name
\end{lstlisting}
\quad To break when a function is called in \code{file.c}
%
\begin{lstlisting}[belowskip=-0.8 \baselineskip]
tbreak <loc>
\end{lstlisting}
\quad Make temporary breakpoint that deletes itself after \\
\quad it gets hit once
%
\begin{lstlisting}[belowskip=-0.8 \baselineskip]
info break
\end{lstlisting}
\quad show info on currently set \code{breakpoints}
%
\begin{lstlisting}[belowskip=-0.8 \baselineskip]
del <brnr>
\end{lstlisting}
\quad delete \code{breakpoint} $<$brnr$>$ (find $<$brnr$>$\\
\quad using \code{info break})
%
\begin{lstlisting}[belowskip=-0.8 \baselineskip]
disable/enable <brnr>
\end{lstlisting}
\quad disable/enable \code{breakpoint} $<$brnr$>$ (skip or don't skip \\
\quad it without deleting it)
%
\begin{lstlisting}[belowskip=-0.8 \baselineskip]
save breakpoints <filename>
\end{lstlisting}
\quad save breakpoints of current session in $<$filename$>$
%
\begin{lstlisting}[belowskip=-0.8 \baselineskip]
source <filename>
\end{lstlisting}
\quad load (e.g. breakpoints) from $<$filename$>$
%









\gdbsection{gdb and ``value has been optimized out''}

Either recompile your program without optimization (\code{-O0}), or tell
compiler not to optimize specific function you're looking at. Doing that
depends on the compiler.

\begin{lstlisting}[belowskip=-0.8 \baselineskip]
#pragma GCC push_options
#pragma GCC optimize ("O0")
  void your_function(){...}
#pragma GCC pop_options
\end{lstlisting}
%
\quad for \code{GCC}
%
\begin{lstlisting}[belowskip=-0.8 \baselineskip]
#pragma optimize( "", off )
  void your_function() {...}
#pragma optimize( "", on )
\end{lstlisting}
\quad or
\begin{lstlisting}[belowskip=-0.8 \baselineskip]
#pragma intel optimization_level 0
	void your_funtction(){...}
\end{lstlisting}
%
\quad For \code{intel}
%
\begin{lstlisting}[belowskip=-0.8 \baselineskip]
__attribute__((optnone))
	void your_function(){...}
\end{lstlisting}
\quad For \code{clang}







\gdbsection{gdb and MPI}

\begin{lstlisting}[belowskip=-0.8 \baselineskip]
$ mpirun -n 4 xterm -e gdb -ex \
  run your_program
\end{lstlisting}
\quad Runs 4 \code{xterm} terminals with \code{MPI} and\\
\quad executes your program through \code{gdb}

\begin{lstlisting}[belowskip=-0.8 \baselineskip]
$ mpirun -n 4 xterm -e gdb -ex \
  run --args your_program --arg1 --arg2
\end{lstlisting}
\quad same as above, but allows your program to \\
\quad read in command line arguments




\end{document}
