\documentclass[10pt, a4paper, english, parskip, twocolumn]{scrartcl}

\usepackage{color, graphicx, float}
\usepackage[utf8]{inputenc}
\usepackage[english]{babel}

\usepackage{geometry}
\geometry{margin=1.5cm}

\usepackage[document]{ragged2e}
\usepackage{listings}

\lstdefinestyle{myCodeStyle}{
    backgroundcolor=\color{ashgrey!20},
%     commentstyle=\color{codegreen},
%     keywordstyle=\color{magenta},
%     numberstyle=\tiny\color{codegray},
%     stringstyle=\color{codepurple},
    basicstyle=\ttfamily,
    breakatwhitespace=false,
    breaklines=true,
    captionpos=b,
    keepspaces=true,
    numbers=none,
    numbersep=5pt,
    showspaces=false,
    showstringspaces=false,
    showtabs=false,
    tabsize=2
}
\lstset{style=myCodeStyle}



%-----------------------------------------------
% FORMAT TITLE
%-----------------------------------------------

% Set fonts of document parts
\setkomafont{title}{\rmfamily\bfseries\boldmath}
\addtokomafont{section}{\rmfamily\bfseries\boldmath}
\addtokomafont{subsection}{\rmfamily\bfseries\boldmath}
\addtokomafont{subsubsection}{\rmfamily\bfseries\boldmath}
\addtokomafont{disposition}{\rmfamily} % table of contents and stuff
\setkomafont{descriptionlabel}{\rmfamily\bfseries\boldmath}






%-----------------------------------------------
% TEXT IN BOXES
%-----------------------------------------------

\usepackage[framemethod=TikZ]{mdframed}

% New Colors (needs to be after usepackage mdframed)
\definecolor{babyblueeyes}{rgb}{0.63, 0.79, 0.95}
\definecolor{ashgrey}{rgb}{0.7, 0.75, 0.71}
\definecolor{caribbeangreen}{rgb}{0.0, 0.8, 0.6}
\definecolor{bittersweet}{rgb}{1.0, 0.44, 0.37}

\mdfdefinestyle{section}{%
       %rightline=true,
       innerleftmargin=10,
       innerrightmargin=10,
       frametitlerule=false,
%        frametitlerulecolor=black,
%        frametitlerulewidth=2
}





\newcommand{\code}[1]{\texttt{#1}}
\newcommand{\ddash}{-{}-}
\newcommand{\gdbsection}[1]{\begin{mdframed}[frametitlebackgroundcolor=bittersweet,style=section,
frametitle=#1]\end{mdframed}}



\pagestyle{empty}


\begin{document}
	

\section*{gdb Debugging Cheatsheet}




\gdbsection{running gdb}

Remember to compile your program with debugging symbols enabled!
That's usually the \code{-g} or \code{-ggdb} flag.

Running your executable with \code{gdb}:
%
\begin{lstlisting}[belowskip=0pt]
$ gdb /path/to/your/executable
\end{lstlisting}
%
\begin{lstlisting}[belowskip=-0.8 \baselineskip]
$ gdb --args /path/to/your/executable --arg1 --arg2
\end{lstlisting}
%
\quad\quad If your executable requires command line \\
\quad\quad arguments (\code{\ddash arg1 \ddash arg2})







\gdbsection{gdb basic commands}
%
{\renewcommand{\arraystretch}{1.15}%
% \small%
\begin{table}[H]
\begin{center}
\begin{tabular}{p{2.8cm}p{5.4cm}}
\code{r (run)}          & run the program \code{gdb} loaded \\
\code{c (continue)}     & continue run\\
\code{n (next)}         & execute next line\\
\code{s (step)}         & execute next step\\
\code{finish}           & after a breakpoint, run function until end and halt there\\[1em]
%
\code{l (list)}         & show source code at current loc\\
\code{l <lnr>}          & show source code at line <lnr>\\
\code{tui enable}       & start a fancy Text User Interface\\[1em]
%
\code{p (print) <var>}  & print variable <var>\\
\code{p *<var\_p>}      & print value of pointer <var\_p> instead of address\\
\code{display <var>}    & print variable <var> every time it is touched throughout the run\\
\code{info locals}      & print all local variables\\[1em]
%
\code{b (break) <loc>}  & set a breakpoint at <loc>\\
\code{tbreak <loc>}     & set a temporary breakpt at <loc>\\
\code{watch <var>}      & set a watchpoint at var <var>\\[1em]
%
\code{bt (backtrace)}   & show stack trace\\
\code{where}            & show current location in trace\\
\code{frame}            & show current location in trace\\
\code{frame <nr>}       & change into frame <nr>
\end{tabular}
\end{center}
\end{table}
}







\gdbsection{gdb and MPI}

\begin{lstlisting}[belowskip=-0.8 \baselineskip]
$ mpirun -n 4 xterm -e gdb -ex run your_program
\end{lstlisting}
\quad\quad Runs 4 \code{xterm} terminals with \code{MPI} and\\
\quad\quad executes your program through \code{gdb}

\begin{lstlisting}[belowskip=-0.8 \baselineskip]
$ mpirun -n 4 xterm -e gdb -ex
    run --args your_program --arg1 --arg2
\end{lstlisting}
\quad\quad same as above, but allows your program to \\
\quad\quad read in command line arguments







\gdbsection{gdb and core dumps}

\begin{lstlisting}[belowskip=-0.8 \baselineskip]
$ ulimit -S -c unlimited
\end{lstlisting}
\quad\quad To enable core dumps on linux

\begin{lstlisting}[belowskip=-0.8 \baselineskip]
$ coredump
\end{lstlisting}
\quad\quad Command shows you where cores will be \\
\quad\quad dumped on your system

\begin{lstlisting}[belowskip=-0.8 \baselineskip]
$ gdb -c core.XXXX path/to/executable
\end{lstlisting}
\quad\quad Load core dump with \code{gdb}









\gdbsection{gdb breakpoints}

\begin{lstlisting}[belowskip=-0.8 \baselineskip]
break <line number in main file>
\end{lstlisting}
%
\quad To set a breakpoint (\textit{before} you execute \code{run})
%
\begin{lstlisting}[belowskip=-0.8 \baselineskip]
break path/to/file.c:<line_nr>
\end{lstlisting}
\quad To set a breakpoint on <line\_nr> in specific file
%
\begin{lstlisting}[belowskip=-0.8 \baselineskip]
break file.c:function_name
\end{lstlisting}
\quad To break when a function is called in \code{file.c}
%
\begin{lstlisting}[belowskip=-0.8 \baselineskip]
tbreak <loc>
\end{lstlisting}
\quad Make temporary breakpoint that deletes itself after \\
\quad it gets hit once
%
\begin{lstlisting}[belowskip=-0.8 \baselineskip]
info break
\end{lstlisting}
\quad show info on currently set \code{breakpoints}
%
\begin{lstlisting}[belowskip=-0.8 \baselineskip]
del <brnr>
\end{lstlisting}
\quad delete \code{breakpoint} <brnr> (find <brnr>\\
\quad using \code{info break})
%
\begin{lstlisting}[belowskip=-0.8 \baselineskip]
disable/enable <brnr>
\end{lstlisting}
\quad disable/enable \code{breakpoint} <brnr> (skip or don't skip \\
\quad it without deleting it)
%









\gdbsection{gdb and ``value has been optimized out''}

Either recompile your program without optimization, or tell compiler not to optimize specific
function you're looking at. Doing that depends on the compiler.


For \code{GCC}:
\begin{lstlisting}[belowskip=-0.8 \baselineskip]
#pragma GCC push_options
#pragma GCC optimize ("O0")

  void your_function(){...}

#pragma GCC pop_options
\end{lstlisting}
%

For \code{intel}:
\begin{lstlisting}[belowskip=-0.8 \baselineskip]
#pragma optimize( "", off )

  void your_function() {...}

#pragma optimize( "", on )
\end{lstlisting}
\quad\quad or
\begin{lstlisting}[belowskip=-0.8 \baselineskip]
#pragma intel optimization_level 0
	void your_funtction(){...}
\end{lstlisting}
%

For \code{clang}:
\begin{lstlisting}[belowskip=-0.8 \baselineskip]
__attribute__((optnone))
	void your_function()
    {...}
\end{lstlisting}



\end{document}
